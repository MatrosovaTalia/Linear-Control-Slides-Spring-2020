\documentclass{beamer}
\usetheme{Madrid}


 \pdfmapfile{+sansmathaccent.map}
 
\renewcommand{\familydefault}{\rmdefault}


\usepackage{amsmath}
\usepackage{mathtools}

\DeclareMathOperator*{\argmin}{arg\,min}

\title{Observer Design, Kalman Filter}
\author{by Sergei Savin}
\centering
\date{Spring 2020}


\begin{document}
\maketitle


\begin{frame}{Content}
\begin{itemize}
\item Observer Design
\begin{itemize}
\item Observer as a controllable LTI
\item Estimation error dynamics
\item State estimation error stability
\item Observer Design vs Controller design
\item Special case
\item General case: design via Riccati eq.
\end{itemize}

\item  Kalman Filter
\begin{itemize}
\item System with noise and disturbances
\item Estimates definitions
\item Estimate updates
\item Knows and unknowns
\item Covariance update and Kalman gain
\end{itemize}
\end{itemize}
\end{frame}

\begin{frame}{Observer Design}
\framesubtitle{Observer as a controllable LTI}
\begin{flushleft}


\begin{itemize}
\item From the previous lecture we remember that we wanted the estimate of the state to have the same dynamics as the actual state.
\item We also remember that we want to take into account the error between the estimated and measured output $\mathbf y$
\end{itemize}

\bigskip

Assume the measurement is perfect: $\hat{\mathbf y} = \mathbf y$. Then we can propose observer as the following dynamical system:

\[
\hat{\dot {\mathbf x}} = \mathbf A \hat{\mathbf x} + \mathbf B \mathbf u + \mathbf L(\mathbf y - \hat{\mathbf y})
\]

\end{flushleft}
\end{frame}

\begin{frame}{Observer Design}
\framesubtitle{Estimation error dynamics}
\begin{flushleft}

Remember that state estimation error is $\epsilon = \hat{\mathbf x} - \mathbf x$ and the actual dynamics of the system is:
%
\[
\begin{cases}
\dot {\mathbf x} = \mathbf A \mathbf x + 
\mathbf B \mathbf u \\
\mathbf y = \mathbf C \mathbf x 
\end{cases}
\]
%
Then, we can write the state estimation error dynamics:
%
\[
\hat{\dot {\mathbf x}} - \dot {\mathbf x} = 
\mathbf A \hat{\mathbf x} - \mathbf A \mathbf x
+ \mathbf B \mathbf u - \mathbf B \mathbf u 
+ \mathbf L(\mathbf y - \hat{\mathbf y})
\]
%
or:
%
\[
\dot {\epsilon} = \mathbf A \epsilon + 
\mathbf L(\mathbf C \mathbf x - \mathbf C \hat{\mathbf x})
\]
%
or:
%
\[
\dot {\epsilon} = \mathbf A \epsilon - 
\mathbf L \mathbf C \epsilon
\]

\end{flushleft}
\end{frame}

\begin{frame}{Observer Design}
\framesubtitle{State estimation error stability}
\begin{flushleft}

Thus, with the proposed observer, state estimation error is:

\[
\dot {\epsilon} = (\mathbf A - 
\mathbf L \mathbf C) \epsilon
\]

From which immediately follows that the observer is \emph{stable} (i.e., the state estimation error tends to zero), as long as the following matrix is negative-definite:

\[
\mathbf A - 
\mathbf L \mathbf C < 0
\]

We need to find $\mathbf L$.

\end{flushleft}
\end{frame}



\begin{frame}{Observer Design}
\framesubtitle{Observer Design vs Controller design}
\begin{flushleft}

Let us observe the key difference between observer design and controller design.

\bigskip

Controller design: find such $\mathbf K$ that:

\[
\mathbf A - 
\mathbf B \mathbf K < 0
\]

Observer design: find such $\mathbf L$ that:

\[
\mathbf A - 
\mathbf L \mathbf C < 0
\]

We have instruments for finding $\mathbf K$, what about $\mathbf L$?

\end{flushleft}
\end{frame}


\begin{frame}{Observer Design}
\framesubtitle{Special case}
\begin{flushleft}

Assume that $\mathbf C$ is identity $\mathbf I$. Then the observer design problem becomes: 

\begin{itemize}
    \item find such $\mathbf L$ that $\mathbf A - \mathbf L \mathbf I < 0$.
\end{itemize}

Which is equivalent to the problem: 

\begin{itemize}
    \item find such $\mathbf L$ that $\mathbf A - \mathbf B  \mathbf L < 0$, where $\mathbf B = \mathbf I$.
\end{itemize}

And in that formulation, it is equivalent to the controller design problem, which we know how to solve.

\end{flushleft}
\end{frame}


\begin{frame}{Observer Design}
\framesubtitle{General case: design via Riccati eq.}
\begin{flushleft}

In general, we can observe that if $\mathbf A - \mathbf L \mathbf C$ is negative-definite, then $(\mathbf A - 
\mathbf L \mathbf C)^{\top}$ is negative-definite too (by definition of the negative-definiteness). 

\bigskip

Therefore, we can solve the following \emph{dual problem}:

\begin{itemize}
    \item find such $\mathbf L$ that $\mathbf A^{\top} - 
\mathbf C^{\top} \mathbf L^{\top} < 0$.
\end{itemize}

\bigskip

The dual problem is \emph{equivalent} to the control design problem. We can solve it by producing and solving algebraic Riccati equation, as in the LQR formulation. In pseudo-code it can be represented the following way:

\bigskip

$\mathbf L^{\top}$ \texttt{= lqr}($\mathbf A^{\top}$, $\mathbf C^{\top}$, $\mathbf Q$, $\mathbf R$).

where $\mathbf Q$ and $\mathbf R$ are weight  matrices, determining the "sensitivity" or "aggressiveness" of the observer.


\end{flushleft}
\end{frame}




\begin{frame}{Kalman Filter}
\framesubtitle{System with noise and disturbances}
\begin{flushleft}

Assume we have a discrete linear system with disturbance and noise:

\[
\begin{cases}
\mathbf x_{i+1} = \mathbf A \mathbf x_i + 
\mathbf B \mathbf u_i + \mathbf w\\
\mathbf y_i = \mathbf C \mathbf x_i  + \mathbf v
\end{cases}
\]

where $\mathbf w$ and $\mathbf v$ are disturbance and noise, both are random processes and both assumed to have 0 expected value. Covariance matrix of $\mathbf w$ is $\mathbf W$, covariance matrix of $\mathbf v$ is $\mathbf V$.

\bigskip

Our estimation $\hat{\mathbf x}$ is now done is two steps: after dynamics update, and after sensor update. We will use the following notation for it:

\begin{itemize}
    \item $\hat{\mathbf x}_{i|i-1}$ is the $i$-th step estimate after dynamics update; also called an \emph{a priori} estimate.
    \item $\hat{\mathbf x}_{i|i}$ is the estimate after sensor update; also called an \emph{a posteriori} estimate.
\end{itemize}

\end{flushleft}
\end{frame}


\begin{frame}{Kalman Filter}
\framesubtitle{Estimates covariances}
\begin{flushleft}


Both estimates $\hat{\mathbf x}_{i|i-1}$ and $\hat{\mathbf x}_{i|i}$ have associated state estimation errors:

\begin{itemize}
    \item $\epsilon_{i|i-1} = \mathbf x_i - \hat{\mathbf x}_{i|i-1}$
    \item $\epsilon_{i|i} = \mathbf x_i - \hat{\mathbf x}_{i|i}$
\end{itemize}

\bigskip

Those state estimation errors have their covariance matrices:

\begin{itemize}
    \item $\mathbf P_{i|i-1} = \text{cov}(\epsilon_{i|i-1})$, 
    \item $\mathbf P_{i|i} = \text{cov}(\epsilon_{i|i})$
\end{itemize}

We can consider the output estimation error $\varepsilon_i = \mathbf y_i - \mathbf C \hat{\mathbf x}_{i|i-1}$ and its covariance matrix $\mathbf Y_i$: 

\[
\mathbf Y_i = \text{cov}(\mathbf y_i - \mathbf C \hat{\mathbf x}_{i|i-1})
\]

\end{flushleft}
\end{frame}



\begin{frame}{Kalman Filter}
\framesubtitle{Estimate updates}
\begin{flushleft}


We know that $\hat{\mathbf x}_{i|i-1}$ can be found as follows:

\[
\hat{\mathbf x}_{i|i-1} = \mathbf A \hat{\mathbf x}_{i-1|i-1} + \mathbf B \mathbf u_i
\]

where $\hat{\mathbf x}_{i-1|i-1}$ is the estimate on the previous time step.

\bigskip

We will search for the update law for the a posteriori estimate $\hat{\mathbf x}_{i|i}$ in the following form:

\[
\hat{\mathbf x}_{i|i} = \hat{\mathbf x}_{i|i-1} + 
\mathbf K_i (\mathbf y_i - \mathbf C \hat{\mathbf x}_{i|i-1})
\]

We need to find $\mathbf K_i$, which is the Kalman filter gain.


\end{flushleft}
\end{frame}



\begin{frame}{Kalman Filter}
\framesubtitle{Knows and unknowns}
\begin{flushleft}

We know or measure:

\begin{itemize}
    \item $\mathbf A$, $\mathbf B$ and $\mathbf C$
    \item $\mathbf W$ and $\mathbf V$
    \item $\mathbf y_i$
\end{itemize}

\bigskip

We need to find:

\begin{itemize}
    \item $\hat{\mathbf x}_{i|i-1}$ and $\hat{\mathbf x}_{i|i}$
    \item $\mathbf K_i$
    \item $\mathbf Y_i$, $\mathbf P_{i|i-1}$ and $\mathbf P_{i|i}$
\end{itemize}


\end{flushleft}
\end{frame}



\begin{frame}{Kalman Filter}
\framesubtitle{Covariance update and Kalman gain}
\begin{flushleft}

There is a formula for the a priori estimation covariance update:

\[
\mathbf P_{i|i-1} = 
\mathbf A \mathbf P_{i-1|i-1} \mathbf A^{\top} + \mathbf W
\]

And a formula for the measurement covariance:

\[
\mathbf Y_i = 
\mathbf C \mathbf P_{i|i-1} \mathbf C^{\top} + \mathbf V
\]

With those two, we can find the Kalman filter gain:

\[
\mathbf K_i = 
\mathbf P_{i|i-1} \mathbf C^{\top} \mathbf Y_i^{-1}
\]

Now we can find the a posteriori covariance:

\[
\mathbf P_{i|i} = 
(\mathbf I - \mathbf K_i \mathbf C) \mathbf P_{i|i-1}
\]

\end{flushleft}
\end{frame}



\begin{frame}
\centerline{Lecture slides are available via Moodle.}
\bigskip
\centerline{You can help improve these slides at:}
\centerline{\url{https://github.com/SergeiSa/Linear-Control-Slides-Spring-2020}}
\bigskip
\centerline{Check Moodle for additional links, videos, textbook suggestions.}
\end{frame}

\end{document}
